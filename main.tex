\documentclass[12pt]{ociamthesis}  % default square logo 
%\documentclass[12pt,beltcrest]{ociamthesis} % use old belt crest logo
%\documentclass[12pt,shieldcrest]{ociamthesis} % use older shield crest logo

%load any additional packages
\usepackage{amssymb}
\usepackage{caption}
\usepackage{multirow}
\usepackage{listings}
\usepackage{longtable}

%input macros (i.e. write your own macros file called mymacros.tex 
%and uncomment the next line)
%\include{mymacros}

\title{\large Judul Laporan Tingkat Akhir \\[5ex]   %your thesis title,
\small{Laporan ini dibuat untuk memenuhi persyaratan kelulusan matakuliah\\[-4ex]
	Program Tugas Akhir}} %note \\[1ex] is a line break in the title

\author{\large\textbf{Nama Mahasiswa}}             %your name
\college{\large\textbf{X.XX.X.XXX}\\[5ex]}

%\textit{In Partial Fulfilment of The Requirements for The Degree of Applied Bachelor of Informatics Engineering}}  %your college

%\renewcommand{\submittedtext}{change the default text here if needed}
\degree{\textbf{POLITEKNIK POS INDONESIA}}     %the degree
%\degreedate{Bandung}         %the degree date
\degreedate{\textbf{2019}}  

%end the preamble and start the document
\begin{document}

%this baselineskip gives sufficient line spacing for an examiner to easily
%markup the thesis with comments
\baselineskip=18pt plus1pt

%set the number of sectioning levels that get number and appear in the contents
\setcounter{secnumdepth}{12}
\setcounter{tocdepth}{12}

\maketitle                  % create a title page from the preamble info
\begin{romanpages}
\addcontentsline{toc}{chapter}{COVER}
\sloppy

\begin{covereng}
	
\centering{\textbf{\textit{This Report Submitted to Fulfill The Requirements of Applied Bachelor of Informatics Engineering}}} \\[10ex]

\begin{figure}[!htbp]
	\centering
	\includegraphics[width=32mm]{figures/logo-poltekpos} \\[4ex]
\end{figure}

\centering{\large{\textbf{Nama Mahasiswa}}}

\centering{\large{\textbf{X.XX.X.XXX}}} \\[18ex]

\begin{longtable}{p{1\textwidth}}

\small{\textit{\textbf{APPLIED BACHELOR PROGRAM OF INFORMATICS ENGINEERING}}} \\[2ex]
\centering{\textbf{POLITEKNIK POS INDONESIA}} \\[2ex]
\textbf{BANDUNG} \\[2ex]
\textbf{2019} \\

\end{longtable}

\end{covereng}
\include{section/dedication}   % include a dedication.tex file

\addcontentsline{toc}{chapter}{LEMBAR PENGESAHAN LAPORAN}
\begin{pengesahan}
\thispagestyle{plain}
	
\begin{figure}[!htbp]
	\centering
	\includegraphics[width=32mm]{figures/logo-poltekpos} \\[4ex]
\end{figure} 

\centering\textbf{LEMBAR PENGESAHAN LAPORAN TUGAS AKHIR} \\[2ex]

\centering{\textbf{\textit{Disusun Oleh,}}} \\
\centering{\textbf{\textit{Nama Mahasiswa}}} \\
\centering{\textbf{\textit{NPM}}} \\[2ex]

Bandung, 27 Agustus 2019 \\[2ex]

\begin{table}[h]
	\begin{tabular}{ccc}
		Dosen Pembimbing Utama,                  &                      & Dosen Pembimbing Pendamping,       \\
		&                      &                                    \\
		&                      &                                    \\
		&                      &                                    \\
		&                      &                                    \\
		\underline{Rolly Maulana Awangga, S.T., M.T.}        &                      & \underline{Syafrial Fachri Pane, S.T., M.T.I.} \\
		NIK: 117.86.219                          &                      & NIK: 117.88.233                    \\
		\multicolumn{1}{l}{}                     & \multicolumn{1}{l}{} & \multicolumn{1}{l}{}               \\
		Mengetahui,                              &                      &                                    \\
		Kaprodi D4 Teknik Informatika            &                      & Dosen Penguji,                     \\
		&                      &                                    \\
		&                      &                                    \\
		&                      &                                    \\
		&                      &                                    \\
		\underline{M. Yusril Helmi Setyawan, S.Kom., M.Kom.} &                      & \underline{Rd. Nuraini, Siti Fatonah, S.S., M.Hum.}    \\
		NIK: 113.74.163                          &                      & NIK: 217.72.187                   
	\end{tabular}
\end{table}

\end{pengesahan}

     %include a pengesahan.tex file

\addcontentsline{toc}{chapter}{LEMBAR PERNYATAAN}
\sloppy

\begin{pernyataan}
\thispagestyle{plain}

Dengan ini saya,

\begin{table}[ht]
	\centering
	\begin{tabular}{lll}
		Nama     & : & Nama Mahasiswa                                                                                                                                  \\
		NPM      & : & NPM                                                                                                                                    \\
		Judul TA & : & \begin{tabular}[c]{@{}l@{}} JUDUL \\ LAPORAN TUGAS AKHIR TEKNIK INFORMATIKA / SI \end{tabular}
	\end{tabular}
\end{table}

Memberikan kepada Politeknik Pos Indonesia hak non-eksklusif untuk menyimpan, memperbanyak, dan menyebarluaskan tesis karya saya, secara keseluruhan atau hanya sebagian atau hanya ringkasan saja, dalam bentuk format tercetak dana tau elektronik. \\[2ex]
Menyatakan bahwa saya, akan mempertahankan hak exclusive saya, untuk menggunakan seluruh atau sebagian isi tesis saya, guna pengembangan karya di masa depan, misalnya bentuk artikelm buku, perangkat lunak, ataupun sistem informasi. \\[2ex]
\textit{Hereby grant to my school, Politeknik Pos Indonesia the non-exclusive right to archive, reproduce, and distribute my thesis, in whole or in part whether in the form of printed and electronic formats.} \\[2ex]
\textit{I acknowledge that I retain exclusive rights of my thesis by using all or part of it in the future work or outputs, such as article, book software and information system.} \\[1ex]
\begin{flushright}
Bandung, 27 Agustus 2019 \\[10ex]


Maulyanda
\end{flushright}

\end{pernyataan}    %include a pernyataan.tex file

\addcontentsline{toc}{chapter}{HALAMAN PERNYATAAN}
\sloppy

\begin{halaman}
\thispagestyle{plain}

Saya, nama Mahasiwa, NIM 1154008 menyatakan dengan sebenar-benarnya bahwa Tugas Akhir/Skripsi saya berjudul ``JUDUL LAPORAN TUGAS AKHIR'' adalah merupakan gagasan dan hasil penelitian saya sendiri dengan bimbingan Dosen Pembimbing.
Saya juga menyatakan dengan sebenarnya bahwa isi Tugas Akhir/Skripsi ini tidak merupakan jiplakan dan bukan pula dari karya orang lain, kecuali kutipan dari literatur dana tau hasil wawancara tertulis yang saya acu dan telah saya sebutkan di Daftar Acuan dan Daftar Pustaka.
Demikian pernyataan ini saya buat dengan sebenarnya dan saya bersedia menerima sanksi apabila ternyata pernyataan saya ini tidak benar. \\[2ex]
\textit{I, Name Mahaiswa, Student ID 1154008 truly acknowledge that my thesis with title ``JUDUL LAPORAN TUGAS AKHIR'' is my concept and project result with guidance from supervisor.
I, also truly acknowledge that content of this thesis are not copyed and not from another people work, except my citation from literature or written interview result and already write in reference list and bibliography list. 
That’s my acknowledge were truly made and if in reality this acknowledge weren’t true, I willing sanction.}
\begin{flushright}
Bandung, 27 Agustus 2019\\
Yang menyatakan \\[11ex]

Nama Mahasiswa \\
NPM
\end{flushright}

\end{halaman}      %include a halaman.tex file

\addcontentsline{toc}{chapter}{ABSTRAK}
\include{section/abstrak}  % include the abstract

\addcontentsline{toc}{chapter}{\textit{ABSTRACT}}
\include{section/abstract}

% start roman page numbering
\addcontentsline{toc}{chapter}{KATA PENGANTAR}
\begin{acknowledgements}
\thispagestyle{plain}
Assalamualaikum warahmatullahi wabarakatuh. Segala puji bagi Allah SWT yang telah memberikan kemudahan sehingga dapat menyelesaikan laporan Tugas Akhir ini, tanpa pertolongan-Nya mungkin penulis tidak akan sanggup menyelesaikannya dengan baik. Shalawat dan salam semoga terlimpah curahkan kepada Nabi Muhammad SAW beserta sahabat dan keluarga Beliau.

Laporan ini disusun untuk memenuhi kelulusan matakuliah Tugas Akhir pada Program Studi DIV Teknik Informatika. Proses Tugas Akhir ini juga tidak terlepas dari bantuan berbagai pihak. Oleh karena itu, pada kata pengantar ini penulis menyampaikan teriamakasih kepada :
\begin{enumerate}

\item Rolly Maulana Awangga, S.T., M.T. selaku Pembimbing Utama dalam penyusunan laporan Tugas Akhir ini;
\item	Syafrial Fachri Pane, S.T., M.T.I. selaku Pembimbing Pendamping dalam penyusunan laporan Tugas Akhir dan Koordinator Tugas Akhir Tahun Akademik 2018/2019;
\item	M. Yusril Helmi Setyawan, S.Kom., M.Kom. selaku Ketua Program Studi DIV Teknik Informatika Tahun Akademik 2018/2019;
\item	Dr. Ir. Agus Purnomo, M.T. selaku Direktur Politeknik Pos Indonesia Tahun Akademik 2018/2019.

\end{enumerate}

Penulis telah membuat laporan ini dengan sebaik-baiknya, diharapkan memberikan kritik dan saran dari semua pihak yang bersifat membangun, terimakasih.

\begin{raggedleft}

Bandung, 2019

Penulis

\end{raggedleft}

\end{acknowledgements}   % include an acknowledgements.tex file

\addcontentsline{toc}{chapter}{DAFTAR ISI}
\tableofcontents % generate and include a table of contents

\addcontentsline{toc}{chapter}{DAFTAR GAMBAR}
\listoffigures

\addcontentsline{toc}{chapter}{DAFTAR TABEL}
\listoftables           % generate and include a list of figures

\addcontentsline{toc}{chapter}{DAFTAR SIMBOL}
\include{section/simbol}

\addcontentsline{toc}{chapter}{DAFTAR SINGKATAN}
\begin{singkatan}
\thispagestyle{plain}
\begin{table}[ht]
	\centering
	\begin{tabular}{|c|c|c|}
	\hline
		\textbf{No} & \textbf{Singkatan} & \textbf{Keterangan}   \\
	\hline
		1. & IRC       & \textit{Informatics Research Center}  \\
		2. & Dibuat Oleh & Maulyanda, Resti, dan Ikhsan \\
	\hline    
	\end{tabular}
\end{table}

\end{singkatan}

\end{romanpages}            % end roman page numbering

%now include the files of latex for each of the chapters etc
\include{section/chapter1} 
\include{section/chapter2} 
\chapter{GAMBARAN OBYEK STUDI}

\subsection{Data Primer}

\subsection{Data Sekunder}
 
\include{section/chapter4} 
\include{section/chapter5} 
\chapter{KESIMPULAN}

\section{Kesimpulan Masalah}

\section{Kesimpulan Metode}

\section{Kesimpulan Pengujian Sistem} 
%\include{section/chapter7}

%next line adds the Bibliography to the contents page
\addcontentsline{toc}{chapter}{Daftar Pustaka}
%uncomment next line to change bibliography name to references
%\renewcommand{\bibname}{References}
\bibliography{references}        %use a bibtex bibliography file refs.bib
\bibliographystyle{plain}  %use the plain bibliography style

%now enable appendix numbering format and include any appendices
\appendix
\include{section/appendix1}
\include{section/appendix2}

\end{document}

